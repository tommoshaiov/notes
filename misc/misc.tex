\section*{misc}
\auth{Schwarz} $[0,1]\arr{f}\R$ continuous, $f(x+h)+f(x-h)-2f(x)=o(h^2)$ for all $x\in(0,1)$ implies $f(x)=ax+b$ is affine.\\
\und{proof} wlog $f(0)=f(1)=0$. let $g_\eps(t)=f(t)-\eps t(1-t)$. then 
$\frac{g(t+h)+g(t-h)-2g(t)}{h^2}\to 2\eps$ for $t\in(0,1)$ meaning $g_\eps$ has 
maximum at $0\or 1$. hence $g_\eps\le0$ for all $\eps$ and $f\le0$. similarly $f\ge0
$.\qed\\
\und{claim} $a_n=o(n)$ (weakly) increasing sequence of positive integers 
$\implies \disp\frac{n}{a_n}$ contains all positive integers.\\
\und{proof} as $ka_n$ has the same properties, it suffices to show $a_n=n$ has a 
solution. at $n=1$ we have $\ge$, and at some point $<$. we cannot however have a 
flip from $>$ to $<$ as $a_n>n\implies a_{n+1}\ge n+1$.\qed\\
\auth{Fekete} suppose $a_n$ real with $a_{n+m}\le a_n+a_m$. then 
$a_n/n$ converges to its infimum.\\
\proof fixing $m$, we have $n=mq+r$ with $r<m$ and $a_{n}\le qa_m+a_r$ 
(setting $a_0=0$) so that $a_n/n\le a_m\frac{q}{n}+a_r/n\to a_m/m$. thus 
$\limsup a_n/n\le\inf a_m/m$.\\

\und{claim} a bounded harmonic function $g:\R^n\to\R$ is constant.\\
\und{proof} fix two points $p,q$. then 
$f(p)-f(q)=\disp\frac{\int_{B_R(p)\Del B_r(q)}\pm f}{\vol B_R}$, 
but ${\vol (B_R(p)\Del B_R(q))}=o({\vol B_R})$ is negligble.\qed\\
\und{claim} a bounded harmonic function $g:\Z^n\to\R$ is constant.\\
\und{proof} suppose not. let $r(x)=\Del_{e_1}g(x)=g(x+e_1)-g(x)$. then 
$r(x)+r(x+e_1)+\dots+r(x+ke_1)$ is uniformly bounded. wlog $s=\sup r(x)>0$. fix 
$x_\eps$ with $r(x_\eps)>s-\eps$. since $r(x)\le s$ and 
$\frac{1}{2d}\sum_{x\sim y}r(x)=r(y)$, we get $g(x_\eps+e_1)>s-2d\eps$. 
taking $k$ big and $\eps$ small yields arbitrarily large
$r(x_\eps)+r(x_\eps+e_1)+\dots+r(x_\eps+ke_1)>(k+1)s-\eps(1+2d+\dots+(2d)^k)$.\qed\\
\und{claim} assume $\Z=\disp\bigsqcup_{j\in[k]} a_j+d_j\Z$ is a nontrivial $(k\ge2)$ 
partition of the integers into arithmetic progressions. then the differences $d_j$ are not all distinct.\\\\
\und{proof} wlog we have $\Z_{\ge0}=\disp\bigsqcup a_j+d_j\Z_{\ge0}$. now 
$\disp\frac{1}{1-x}=\sum x^n=\sum_j\sum_m x^{a_j+d_jm}=\sum_j \frac{x^{a_j}}{1-x^{d_j}}$. 
since there is no pole at $e^{2\pi i/\max{d_j}}$, the maximal $d_j$ has to be the difference of at least two progressions.\qed\\\\
